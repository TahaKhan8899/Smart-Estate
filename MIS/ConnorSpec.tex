\documentclass[12pt]{article}

\usepackage{graphicx}
\usepackage{paralist}
\usepackage{amsfonts}
\usepackage{amsmath}
\usepackage{hhline}
\usepackage{booktabs}
\usepackage{multirow}
\usepackage{multicol}
\usepackage{url}

\oddsidemargin -10mm
\evensidemargin -10mm
\textwidth 160mm
\textheight 200mm
\renewcommand\baselinestretch{1.0}

\pagestyle {plain}
\pagenumbering{arabic}

\newcounter{stepnum}

%% Comments

\usepackage{color}

\newif\ifcomments\commentstrue

\ifcomments
\newcommand{\authornote}[3]{\textcolor{#1}{[#3 ---#2]}}
\newcommand{\todo}[1]{\textcolor{red}{[TODO: #1]}}
\else
\newcommand{\authornote}[3]{}
\newcommand{\todo}[1]{}
\fi

\newcommand{\wss}[1]{\authornote{blue}{SS}{#1}}

\title{Smart Estate Specification}
\author{COMPSCI 2XB3 L09 Group 9}

\begin {document}

\maketitle
This Module Interface Specification (MIS) document contains modules, types and
methods for implementing Smart Estate.

\newpage

\section* {StateInfo Type Module}

\subsection*{Module}

StateInfo

\subsection* {Uses}

N/A

\subsection* {Syntax}

\subsubsection* {Exported Constants}

None

\subsubsection* {Exported Types}

StateInfo = ?

\subsubsection* {Exported Access Programs}

\begin{tabular}{| l | l | l | p{5cm} |}
\hline
\textbf{Routine name} & \textbf{In} & \textbf{Out} & \textbf{Exceptions}\\
\hline
new StateInfo & String & StateInfo & none\\
\hline
getState &  & String & none\\
\hline
getHPI & & $\mathbb{R}$ & none\\
\hline
setHPI & $\mathbb{R}$ &  & none\\
\hline
getCrimeRate & & $\mathbb{R}$ & none\\
\hline
setCrimeRate & $\mathbb{R}$ &  & none\\
\hline
getHousingPrice & & $\mathbb{R}$ & none\\
\hline
setHousingPrice & $\mathbb{R}$ &  & none\\
\hline
toString & & String & none\\
\hline
\end{tabular}

\subsection* {Semantics}

\subsubsection* {State Variables}

$state$: String\\
$hpi$: $\mathbb{R}$\\
$crime\_rate$: $\mathbb{R}$\\
$housing\_price$: $\mathbb{R}$

\subsubsection* {State Invariant}

None

\subsubsection* {Assumptions \& Design Decisions}

\begin{itemize}
\item The StateInfo constructor is called for each object instance before any
  other access routine is called for that object.  The constructor can only be
  called once.
\item Once state info is gathered for each StateInfo object methods setHPI, setCrimeRate, and setHousingPrice are only called once.
\end{itemize}

\subsubsection* {Access Routine Semantics}

new StateInfo($s$):
\begin{itemize}
\item transition: $state := s$
\item output: $\mathit{out} := \mathit{self}$
\item exception: none
\end{itemize}

\noindent getState():
\begin{itemize}
\item output: $out := state$
\item exception: none
\end{itemize}

\noindent getHPI():
\begin{itemize}
\item output: $out := hpi$
\item exception: none
\end{itemize}

\noindent setHPI($v$):
\begin{itemize}
\item transition: $hpi := v$
\item exception: none
\end{itemize}

\noindent getCrimeRate():
\begin{itemize}
\item output: $out := crime\_rate$
\item exception: none
\end{itemize}

\noindent setCrimeRate($v$):
\begin{itemize}
\item transition: $crime\_rate := v$
\item exception: none
\end{itemize}

\noindent getHousingPrice():
\begin{itemize}
\item output: $out := housing\_price$
\item exception: none
\end{itemize}

\noindent getHousingPrice($v$):
\begin{itemize}
\item transition: $housing\_price := v$
\item exception: none
\end{itemize}

\noindent toString():
\begin{itemize}
\item output: $\mathit{out} :=$ "$state$: HPI: $hpi$ Crime Rate: $crime\_rate$ Housing Price: $housing\_price$"
\item exception: none
\end{itemize}


\newpage

\section* {PopulateStateInfo Module}

\subsection* {Module}

PopulateStateInfo

\subsection* {Uses}

ReadHPI\\
ReadCrimeRate\\
ReadHousingPrices\\
StateInfo\\

\subsection* {Syntax}

\subsubsection* {Exported Access Programs}

\begin{tabular}{| l | l | l | p{5cm} |}
\hline
\textbf{Routine name} & \textbf{In} & \textbf{Out} & \textbf{Exceptions}\\
\hline
initStates &  &  & none\\
\hline
populateHPI & & & none\\
\hline
populateCrimeRate & & & none\\
\hline
populateHousingPrice & & & none\\
\hline
populateStateInfo & & seq of StateInfo & none\\
\hline
\end{tabular}

\subsection* {Semantics}

\subsubsection* {State Variables}

$states$: seq of StateInfo\\
$state\_names$: seq of String = ["Alabama","Alaska", ... , "Wyoming"]

\subsubsection* {State Invariant}

None

\subsubsection* {Assumptions \& Design Decisions}

\begin{itemize}
\item The result of populateStateInfo must be stored in a StateInfo list of 50 length.
\end{itemize}

\subsubsection* {Access Routine Semantics}

initStates():
\begin{itemize}
\item transition: $states := $($\forall s: $String $| s \in state\_names \;\;.\;\; s = $ StateInfo($s$))
\item exception: none
\end{itemize}

\noindent populateHPI():
\begin{itemize}
\item transition: $states := $\\($\forall i: $int $| 0 \le i \le 50 \;.\; states[i].$setHPI(ReadHPI.read\_data("data/hpi.csv").value()))
\item exception: none
\end{itemize}

\noindent populateCrimeRate():
\begin{itemize}
\item transition: $states := $\\($\forall i: $int $| 0 \le i \le 50 \;.\; states[i].$setCrimeRate(ReadCrimeRate.CRList().value()))
\item exception: none
\end{itemize}

\noindent populateHousingPrice():
\begin{itemize}
\item transition: $states := $\\($\forall i: $int $| 0 \le i \le 50 \;.\; states[i].$setHousingPrice(ReadHousingPrices.\\readPrices("data/housingPrices.csv").value()))
\item exception: none
\end{itemize}

\noindent populateStateInfo():
\begin{itemize}
\item transition: initStates(); populateHPI(); populateCrimeRate(); \\populateHousingPrice();
\item output: $\mathit{out} := states$
\item exception: None
\end{itemize}

\newpage

\section* {Binary Search Module}

\subsection* {Module}

binSearch

\subsection* {Uses}

StateInfo

\subsection* {Syntax}

\subsubsection* {Exported Types}

fieldT = {hpi, crime\_rate, housing\_price}

\subsubsection* {Exported Access Programs}

\begin{tabular}{| l | l | l | p{5cm} |}
\hline
\textbf{Routine name} & \textbf{In} & \textbf{Out} & \textbf{Exceptions}\\
\hline
binSearch &  seq of StateInfo, fieldT, $\mathbb{R}$ &  StateInfo & none\\
\hline
\end{tabular}

\subsection* {Semantics}

\subsubsection* {State Variables}

None

\subsubsection* {State Invariant}

None

\subsubsection* {Assumptions \& Design Decisions}

\begin{itemize}
\item The result of populateStateInfo must be stored in a StateInfo list of 50 length.
\end{itemize}

\subsubsection* {Access Routine Semantics}

initStates():
\begin{itemize}
\item transition: 
\item output: $\mathit{out} :=$
\item exception: none
\end{itemize}

\newpage

<<<<<<< HEAD
=======
\section* {ReadCrimeRate Module}

\subsection* {Module}

ReadCrimeRate

\subsection* {Uses}

Pair

\subsection* {Syntax}

\subsubsection* {Exported Constants}

None

\subsubsection* {Exported Access Programs}

\begin{tabular}{| l | l | l | l |}
\hline
\textbf{Routine name} & \textbf{In} & \textbf{Out} & \textbf{Exceptions}\\
\hline
load\_crime\_data & $s: \mbox{string}$ & ~ & ~\\
\hline
\end{tabular}

\subsection* {Semantics}

\subsubsection* {Environment Variables}

crime\_rate\_data: File listing crime rate data

\subsubsection* {State Variables}

None

\subsubsection* {State Invariant}

None

\subsubsection* {Assumptions}

The input file will match the given specification.

\subsubsection* {Access Routine Semantics}

\noindent load\_crime\_data($s$)
\begin{itemize}
\item transition: read data from the file crime\_rate\_data associated with the string s.
  Use this data to create an array of Pairs, which house the name of a state along with
  the avergae number of violent crimes per capita over 49 years for every 100,000 person.

  The csv file has the following format, where $year$, $population$, total number of $violent\ crime$,
  followed by a breakdown of the number of violent crimes into sub categories including murder, robbery, 
  aggravated assault, etc. which is not used in the computation of the overall project. This is split by a 5 wide
  horizzontal gap separating each state's independent statistics. 

  \begin{equation}
    \begin{array}{cccc}
      year_0, & population_0, & violent_crimes_0, & . . . \\
      year_1, & population_1, & violent_crimes_1, & . . . \\
      year_2, & population_2, & violent_crimes_2, & . . . 
      \\
      ..., & ..., & ..., & ...
      \\
      year_{m-1}, & population_{m-1}, & violent_crimes_{m-1}, & . . . \\
    \end{array}
  \end{equation}

\item exception: FileNotFoundException
\end{itemize}


>>>>>>> 8abfb926d0fb1316a15cc65f2d3530309aa3ea5f
\end{document}