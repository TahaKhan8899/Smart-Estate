\documentclass[12pt]{article}

\usepackage{graphicx}
\usepackage{paralist}
\usepackage{amsfonts}
\usepackage{amsmath}
\usepackage{hhline}
\usepackage{booktabs}
\usepackage{multirow}
\usepackage{multicol}
\usepackage{url}

\oddsidemargin -10mm
\evensidemargin -10mm
\textwidth 160mm
\textheight 200mm
\renewcommand\baselinestretch{1.0}

\pagestyle {plain}
\pagenumbering{arabic}

\newcounter{stepnum}

%% Comments

\usepackage{color}

\newif\ifcomments\commentstrue

\ifcomments
\newcommand{\authornote}[3]{\textcolor{#1}{[#3 ---#2]}}
\newcommand{\todo}[1]{\textcolor{red}{[TODO: #1]}}
\else
\newcommand{\authornote}[3]{}
\newcommand{\todo}[1]{}
\fi

\newcommand{\wss}[1]{\authornote{blue}{SS}{#1}}

\title{Smart Estate Specification}
\author{COMPSCI 2XB3 L09 Group 9}

\begin {document}

\maketitle
This Module Interface Specification (MIS) document contains modules, types and
methods for implementing Smart Estate.

\newpage

\section* {StateInfo Type Module}

\subsection*{Module}

StateInfo

\subsection* {Uses}

N/A

\subsection* {Syntax}

\subsubsection* {Exported Constants}

None

\subsubsection* {Exported Types}

StateInfo = ?\\
fieldT = \{hpi, crime\_rate, housing\_price\}

\subsubsection* {Exported Access Programs}

\begin{tabular}{| l | l | l | p{5cm} |}
\hline
\textbf{Routine name} & \textbf{In} & \textbf{Out} & \textbf{Exceptions}\\
\hline
new StateInfo & String & StateInfo & none\\
\hline
getState &  & String & none\\
\hline
getHPI & & $\mathbb{R}$ & none\\
\hline
setHPI & $\mathbb{R}$ &  & none\\
\hline
getCrimeRate & & $\mathbb{R}$ & none\\
\hline
setCrimeRate & $\mathbb{R}$ &  & none\\
\hline
getHousingPrice & & $\mathbb{R}$ & none\\
\hline
setHousingPrice & $\mathbb{R}$ &  & none\\
\hline
toString & & String & none\\
\hline
\end{tabular}

\subsection* {Semantics}

\subsubsection* {State Variables}

$state$: String\\
$hpi$: $\mathbb{R}$\\
$crime\_rate$: $\mathbb{R}$\\
$housing\_price$: $\mathbb{R}$

\subsubsection* {State Invariant}

None

\subsubsection* {Assumptions \& Design Decisions}

\begin{itemize}
\item The StateInfo constructor is called for each object instance before any
  other access routine is called for that object.  The constructor can only be
  called once.
\item Once state info is gathered for each StateInfo object methods setHPI, setCrimeRate, and setHousingPrice are only called once.
\end{itemize}

\subsubsection* {Access Routine Semantics}

new StateInfo($s$):
\begin{itemize}
\item transition: $state := s$
\item output: $\mathit{out} := \mathit{self}$
\item exception: none
\end{itemize}

\noindent getState():
\begin{itemize}
\item output: $out := state$
\item exception: none
\end{itemize}

\noindent getHPI():
\begin{itemize}
\item output: $out := hpi$
\item exception: none
\end{itemize}

\noindent setHPI($v$):
\begin{itemize}
\item transition: $hpi := v$
\item exception: none
\end{itemize}

\noindent getCrimeRate():
\begin{itemize}
\item output: $out := crime\_rate$
\item exception: none
\end{itemize}

\noindent setCrimeRate($v$):
\begin{itemize}
\item transition: $crime\_rate := v$
\item exception: none
\end{itemize}

\noindent getHousingPrice():
\begin{itemize}
\item output: $out := housing\_price$
\item exception: none
\end{itemize}

\noindent getHousingPrice($v$):
\begin{itemize}
\item transition: $housing\_price := v$
\item exception: none
\end{itemize}

\noindent toString():
\begin{itemize}
\item output: $\mathit{out} :=$ "$state$: HPI: $hpi$ Crime Rate: $crime\_rate$ Housing Price: $housing\_price$"
\item exception: none
\end{itemize}


\newpage

\newpage

\section* {Pair Module}

\subsection*{Module}

Pair

\subsection* {Uses}

N/A

\subsection* {Syntax}

\subsubsection* {Exported Constants}

None

\subsubsection* {Exported Types}

Pair : (string, $\mathbb{R}$)

\subsubsection* {Exported Access Programs}

\begin{tabular}{| l | l | l | p{5cm} |}
\hline
\textbf{Routine name} & \textbf{In} & \textbf{Out} & \textbf{Exceptions}\\
\hline
new Pair & String, $\mathbb{R}$ &  & none\\
\hline
state &  & String & none\\
\hline
val &  & $\mathbb{R}$ & none\\
\hline
toString &  & String & none\\
\hline
\end{tabular}

\subsection* {Semantics}

\subsubsection* {State Variables}

$key$: String\\
$val$: $\mathbb{R}$\\

\subsubsection* {State Invariant}

None

\subsubsection* {Assumptions \& Design Decisions}

\begin{itemize}
\item The Pair constructor is called for each object instance before any
  other access routine is called for that object.  The constructor can only be
  called once.
\item The Pair ADT acts as a tuple which holds two items, the state, and some information corresponding to that state.
\end{itemize}

\subsubsection* {Access Routine Semantics}

new Pair($state$, $val$):
\begin{itemize}
\item transition: $key := state$
\item transition: $val := val$
\item exception: none
\end{itemize}

\noindent state():
\begin{itemize}
\item output: $out := key$
\item exception: none
\end{itemize}

\noindent val():
\begin{itemize}
\item output: $out := val$
\item exception: none
\end{itemize}

\noindent toString():
\begin{itemize}
\item output: $out := $ "State: " + this.key + ", Value: " + this.val
\item exception: none
\end{itemize}


\newpage

\section* {Integration Module}

\subsection*{Module}

Integration

\subsection* {Uses}

Window

\subsection* {Syntax}

\subsubsection* {Exported Constants}

None

\subsubsection* {Exported Types}

N/A

\subsubsection* {Exported Access Programs}

\begin{tabular}{| l | l | l | p{5cm} |}
\hline
\textbf{Routine name} & \textbf{In} & \textbf{Out} & \textbf{Exceptions}\\
\hline
run\_Smart\_Estate &   & Window & none\\
\hline
\end{tabular}

\subsection* {Semantics}

\subsubsection* {State Variables}

N/A

\subsubsection* {State Invariant}

None

\subsubsection* {Assumptions \& Design Decisions}

\begin{itemize}
\item This is the main module that ties all aspects of the application together. It generates a Window object, which outputs a GUI to the screen.
\end{itemize}

\subsubsection* {Access Routine Semantics}

run\_Smart\_Estate():
\begin{itemize}
\item out: new Window()
\end{itemize}

\newpage

\section* {PopulateStateInfo Module}

\subsection* {Module}

PopulateStateInfo

\subsection* {Uses}

ReadHPI\\
ReadCrimeRate\\
ReadHousingPrices\\
StateInfo

\subsection* {Syntax}

\subsubsection* {Exported Access Programs}

\begin{tabular}{| l | l | l | p{5cm} |}
\hline
\textbf{Routine name} & \textbf{In} & \textbf{Out} & \textbf{Exceptions}\\
\hline
populateStateInfo & & seq of StateInfo & none\\
\hline
\end{tabular}

\subsection* {Semantics}

\subsubsection* {State Variables}

$states$: seq of StateInfo\\
$state\_names$: seq of String = ["Alabama","Alaska", ... , "Wyoming"]

\subsubsection* {State Invariant}

None

\subsubsection* {Assumptions \& Design Decisions}

\begin{itemize}
\item The result of populateStateInfo must be stored in a StateInfo list of 50 length.
\end{itemize}

\subsubsection* {Access Routine Semantics}

\noindent populateStateInfo():
\begin{itemize}
\item transition: initStates(); populateHPI(); populateCrimeRate(); \\populateHousingPrice();
\item output: $\mathit{out} := states$
\item exception: None
\end{itemize}

\subsubsection* {Local Functions}

\noindent
$\text{initStates}() \equiv$ $states := $($\forall s: $String $| s \in state\_names \;\;.\;\; s = $ StateInfo($s$))\\

\noindent
$\text{populateHPI}() \equiv$ $states := $\\($\forall i: $int $| 0 \le i \le 50 \;.\; states[i].$setHPI(ReadHPI.read\_data("data/hpi.csv").value()))\\

\noindent
$\text{populateCrimeRate}() \equiv$ $states := $\\($\forall i: $int $| 0 \le i \le 50 \;.\; states[i].$setCrimeRate(ReadCrimeRate.CRList().value()))\\

\noindent
$\text{populateHousingPrice}() \equiv$ $states := $\\($\forall i: $int $| 0 \le i \le 50 \;.\; states[i].$setHousingPrice(ReadHousingPrices.\\readPrices("data/housingPrices.csv").value()))



\newpage

\section* {Binary Search Module}

\subsection* {Module}

binSearch

\subsection* {Uses}

StateInfo\\
Sort

\subsection* {Syntax}

\subsubsection* {Exported Access Programs}

\begin{tabular}{| l | l | l | p{5cm} |}
\hline
\textbf{Routine name} & \textbf{In} & \textbf{Out} & \textbf{Exceptions}\\
\hline
binSearch &  seq of StateInfo, fieldT, $\mathbb{R}$ &  StateInfo & none\\
\hline
binSearch &  seq of StateInfo, String &  StateInfo & none\\
\hline
\end{tabular}

\subsection* {Semantics}

\subsubsection* {Assumptions \& Design Decisions}

\begin{itemize}
\item 
\end{itemize}

\subsubsection* {Access Routine Semantics}

binSearch($arr, field, key$):
\begin{itemize}
\item output: $\mathit{out} := arr[i]$ such that\\
\indent
$isSorted(arr, field) \land $\\ 
$(key \in \{arr.field\} \implies arr[i].field = key) \land$\\
$(key <$ min($\{arr.field\}$)$ \implies arr[i] = arr[0]) \land$\\
$(key >$ max($\{arr.field\}$)$ \implies arr[i] = arr[arr.length-1]) \land$\\
$(key \notin \{arr.field\} \implies arr[i-1].field < arr[i].field = key < arr[i+1].field)$
\item exception: none
\end{itemize}

\noindent
binSearch($arr, key$):
\begin{itemize}
\item output: $\mathit{out} := arr[i]$ such that $arr[i].state = key$
\item exception: none
\end{itemize}

\newpage


\section* {ReadCrimeRate Module}

\subsection* {Module}

ReadCrimeRate

\subsection* {Uses}

Pair

\subsection* {Syntax}

\subsubsection* {Exported Constants}

None

\subsubsection* {Exported Access Programs}

\begin{tabular}{| l | l | l | l |}
\hline
\textbf{Routine name} & \textbf{In} & \textbf{Out} & \textbf{Exceptions}\\
\hline
load\_crime\_data & $s: \mbox{string}$ & ~ & ~\\
\hline
\end{tabular}

\subsection* {Semantics}

\subsubsection* {Environment Variables}

crime\_rate\_data: File listing crime rate data

\subsubsection* {State Variables}

None

\subsubsection* {State Invariant}

None

\subsubsection* {Assumptions}

The input file will match the given specification.

\subsubsection* {Access Routine Semantics}

\noindent load\_crime\_data($s$)
\begin{itemize}
\item transition: read data from the file crime\_rate\_data associated with the string s.
  Use this data to create an array of Pairs, which house the name of a state along with
  the average number of violent crimes per capita over 49 years for every 100,000 person.

  The csv file has the following format, where $year$, $population$, total number of $violent\ crime$,
  followed by a breakdown of the number of violent crimes into sub categories including murder, robbery, 
  aggravated assault, etc. which is not used in the computation of the overall project. This is split by a 5 wide
  horizontal gap separating each state's independent statistics. 

  \begin{equation}
    \begin{array}{cccc}
      year_0, & population_0, & violent_crimes_0, & . . . \\
      year_1, & population_1, & violent_crimes_1, & . . . \\
      year_2, & population_2, & violent_crimes_2, & . . . 
      \\
      ..., & ..., & ..., & ...
      \\
      year_{m-1}, & population_{m-1}, & violent_crimes_{m-1}, & . . . \\
    \end{array}
  \end{equation}

\item exception: FileNotFoundException
\end{itemize}
\newpage

\newpage


\section* {ReadHousingPrices Module}

\subsection* {Module}

ReadHousingPrices

\subsection* {Uses}

Pair

\subsection* {Syntax}

\subsubsection* {Exported Constants}

None

\subsubsection* {Exported Access Programs}

\begin{tabular}{| l | l | l | l |}
\hline
\textbf{Routine name} & \textbf{In} & \textbf{Out} & \textbf{Exceptions}\\
\hline
readPrices & $s: \mbox{string}$ & ~ & ~\\
\hline
\end{tabular}

\subsection* {Semantics}

\subsubsection* {Environment Variables}

fileName: File listing crime rate data

\subsubsection* {State Variables}

None

\subsubsection* {State Invariant}

None

\subsubsection* {Assumptions}

The input file will match the given specification.

\subsubsection* {Access Routine Semantics}

\noindent load\_crime\_data($s$)
\begin{itemize}
\item transition: read data from the file housingPrices.csv associated with the string s.
  Use this data to create an array of Pairs, which contains the name of a state along with
  the median housing price for that particular state.

  The csv file has the following format, where $state$, $price$, and $uncertainty$, is each line in the file. For the purposes of this module, the uncertainty column in not necessary as using the standard price is sufficient.

  \begin{equation}
    \begin{array}{cccc}
      state_0, & price_0, & uncertainty_0, & . . . \\
      state_1, & price_1, & uncertainty_1, & . . . \\
      state_2, & price_2, & uncertainty_2, & . . . 
      \\
      ..., & ..., & ..., & ...
      \\
      state_{m-1}, & price_{m-1}, & uncertainty_{m-1}, & . . . \\
    \end{array}
  \end{equation}

\item exception: FileNotFoundException
\end{itemize}
\newpage

\subsection*{Module}

Sort

\subsection* {Uses}

binSearch
BreadthFirstSearch
Integration

\subsection* {Syntax}

\subsubsection* {Exported Constants}

None

\subsubsection* {Exported Types}

None

\subsubsection* {Exported Access Programs}

\begin{tabular}{| l | l | l | p{5cm} |}
\hline
\textbf{Routine name} & \textbf{In} & \textbf{Out} & \textbf{Exceptions}\\
\hline
sort & sequence of StateInfo, fieldT &  & none\\
\hline
sort &  sequence of StateInfo & & none\\
\hline
\end{tabular}

\subsection* {Semantics}

\subsubsection* {State Variables}
None

\subsubsection* {State Invariant}

None

\subsubsection* {Assumptions \& Design Decisions}

\begin{itemize}
\item Sort is called on a sequence of StateInfo and category by which the sequence is sorted is also indicated when sort is called.
\end{itemize}

\subsubsection* {Access Routine Semantics}

\noindent sort(a, intent) :
\begin{itemize}
\item transition: $a := sort(a, 0, |a| - 1, intent)$
\item exception: none
\end{itemize}

\noindent sort(a):
\begin{itemize}
\item transition: $a := sort(a, 0, |a| - 1)$
\item exception: none
\end{itemize}

\subsubsection* {Local Functions}

\noindent
$\text{sort}(a, lo, hi, intent) \equiv $a := ($\forall i L $T$ | i \in a : a[i - 1]  $<$ a[i])  $ \\


\noindent
$\text{sort}(a, lo, hi) \equiv $a := ($\forall i L $T$ | i \in a : a[i - 1]  $<$ a[i])  $ \\

\noindent
$\text{partition}(a, lo, hi, intent) \equiv j \implies a[lo] $<=$ a[j] $<=$ a[hi]$ \\

\noindent
$\text{partition}(a, lo, hi) \equiv j \implies a[lo] $<=$ a[j] $<=$ a[hi]$ \\

\noindent
$\text{less}(v,  w, intent) \equiv v < w$ \\
\noindent
$\text{less}(v,  w) \equiv v < w$ \\

\noindent
$\text{exch}(a, i, j) \equiv a[i], a[j] := a[j], a[i]$ \\




\newpage
\section* {Graph Type Module}

\subsection*{Module}

Graph

\subsection* {Uses}

BreadthFirstSearch\\
Integration 

\subsection* {Syntax}

\subsubsection* {Exported Constants}
None



\subsubsection* {Exported Types}
None

\subsubsection* {Exported Access Programs}

\begin{tabular}{| l | l | l | p{5cm} |}
\hline
\textbf{Routine name} & \textbf{In} & \textbf{Out} & \textbf{Exceptions}\\
\hline
new Graph & $\mathbb{N}$ &  & none\\
\hline
genGraph & & $Graph$ & none\\
\hline
V & $\mathbb{N}$ &  & none\\
\hline
E & $\mathbb{N}$ &  & none\\
\hline
addEdge & String, String &  & none\\
\hline
adj & $\mathbb{N}$ & sequence of $Bag<String>$ & none\\
\hline
\end{tabular}

\subsection* {Semantics}

\subsubsection* {State Variables}

\begin{itemize}
\item $V$: $\mathbb{N}$
\item $states$: sequence of Strings
\end{itemize}

\subsubsection* {State Invariant}

None

\subsubsection* {Assumptions \& Design Decisions}

\begin{itemize}
\item In order to generate a graph the represents the US, genGraph is called instead of the constructor.
\end{itemize}

\subsubsection* {Access Routine Semantics}

new Graph($V$):
\begin{itemize}
\item transition: $V, E := V, 0$
\item exception: none
\end{itemize}

\noindent genGraph():
\begin{itemize}
\item output: $out :=$ Graph representing the US map
\item exception: none
\end{itemize}

\noindent V():
\begin{itemize}
\item output: $out := V$
\item exception: none
\end{itemize}

\noindent E():
\begin{itemize}
\item output: $out := E$
\item exception: none
\end{itemize}



\noindent addEdge($v, w$):
\begin{itemize}
\item transition: $E, adj[v.index()] := E++, adj[v.index()].add(w)$
\item exception: none
\end{itemize}



\noindent adj($v$):
\begin{itemize}
\item output: $out := adj[v]$
\item exception: none
\end{itemize}



\newpage
\section* {Breadth First Search Module}

\subsection*{Module}

BreadthFirstSearch

\subsection* {Uses}

Integration 

\subsection* {Syntax}
\subsubsection* {Exported Constants}
None



\subsubsection* {Exported Types}
None

\subsubsection* {Exported Access Programs}

\begin{tabular}{| l | l | l | p{5cm} |}
\hline
\textbf{Routine name} & \textbf{In} & \textbf{Out} & \textbf{Exceptions}\\
\hline
BreadthFirstSearch & $Graph, String$ & $sequence of strings$ & none\\
\hline
bfs & & $Graph, String$ $sequence of strings$ & none\\
\hline
getStateInfo & $sequence of strings$ & $sequence of strings$ & none\\
\hline
neighbourStates & String & $sequence of strings$ & none\\
\hline
\end{tabular}

\subsection* {Semantics}

\subsubsection* {State Variables}

\begin{itemize}
\item $path$: $sequence of strings$
\end{itemize}

\subsubsection* {State Invariant}

None

\subsubsection* {Assumptions \& Design Decisions}

\begin{itemize}
\item In order to implement breadth first search on a graph representing the U.S. neighbourStates is called
\end{itemize}

\subsubsection* {Access Routine Semantics}

BreadthFirstSearch($G$, $s$) :
\begin{itemize}
\item output: $out := $ sequence of strings
\item exception: none
\end{itemize}

\noindent bfs($G$, $s$) :
\begin{itemize}
\item output: $out := $ sequence of strings
\item exception: none
\end{itemize}

\noindent getStateInfo($neighbours$) :
\begin{itemize}
\item output: $out := $ sequence of strings
\item exception: none
\end{itemize}

\noindent neighbourStates($startState$):
\begin{itemize}
\item output: $out := $ sequence of strings
\item exception: none
\end{itemize}


\end{document}
